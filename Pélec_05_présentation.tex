\documentclass[8pt,aspectratio=169,hyperref={unicode=true}]{beamer}

\usefonttheme{serif}
\usepackage{fontspec}
	\setmainfont{TeX Gyre Heros}
\usepackage{unicode-math}
\usepackage{lualatex-math}
	\setmathfont{TeX Gyre Termes Math}
\usepackage{polyglossia}
\setdefaultlanguage[frenchpart=false]{french}
\setotherlanguage{english}
%\usepackage{microtype}
\usepackage[locale = FR,
            separate-uncertainty,
            multi-part-units = single,
            range-units = single]{siunitx}
	\DeclareSIUnit\an{an}
  \DeclareSIUnit{\octet}{o}
\usepackage{amsmath}
\usepackage{amsfonts}
\usepackage{amssymb}
\usepackage{array}
\usepackage{graphicx}
\graphicspath{{./Figures/}}
\usepackage{booktabs}
\usepackage{tabularx}
\usepackage{multirow}
\usepackage{multicol}
    \newcolumntype{L}{>{\raggedright\arraybackslash}X}
    \newcolumntype{R}{>{\raggedleft\arraybackslash}X}
\usepackage{tikz}
\usetikzlibrary{mindmap}
\usetikzlibrary{overlay-beamer-styles}
\usepackage{subcaption}
\usepackage[]{animate}
\usepackage{float}
\usepackage{csquotes}

\usetheme[secheader
         ]{Boadilla}
\usecolortheme{seagull}
\setbeamertemplate{enumerate items}[default]
\setbeamertemplate{itemize items}{-}
\setbeamertemplate{navigation symbols}{}
\setbeamertemplate{bibliography item}{}
\setbeamerfont{framesubtitle}{size=\large}
\setbeamertemplate{section in toc}[sections numbered]
%\setbeamertemplate{subsection in toc}[subsections numbered]

\title[Anticipez les besoins en consommation électrique de bâtiments]
{Projet 4 : Anticipez les besoins en consommation électrique de bâtiments}
\author[Lancelot \textsc{Leclercq}]{Lancelot \textsc{Leclercq}} 
\institute[]{}
\date[]{\small{15 décembre 2021}}

\AtBeginSection[]{
  \begin{frame}
  \vfill
  \centering
    \usebeamerfont{title}\insertsectionhead\par%
  \vfill
  \end{frame}
}

\begin{document}
\setbeamercolor{background canvas}{bg=gray!20}
\begin{frame}[plain]
  \titlepage
\end{frame}

\begin{frame}{Sommaire}
  \Large
  \begin{columns}
    \begin{column}{.7\textwidth}
      \tableofcontents[hideallsubsections]
    \end{column}
  \end{columns}
\end{frame}


\section{Introduction}
\subsection{Problématique}
\begin{frame}{\insertsubsection}
  \begin{columns}
    \begin{column}{.6\textwidth}
      \begin{itemize}
        \item Objectif de la ville de Seattle : atteindre la neutralité en émissions
              de carbone
        \item[]
        \item La ville s'intéresse aux émissions des batiments non destinés
              à l'habitation
        \item[]
        \item Pour cela des relevés de consomation ont été réalisés mais ils sont
              couteux à obtenir
        \item[]
        \item Est-il possible de prédire les émissions et de la consommation d'énergie
              pour des batiments pour lesquels les relevés n'ont pas été réalisé à partir
              des relevés déjà obtenus
      \end{itemize}
    \end{column}
    \begin{column}{.4\textwidth}
      \begin{figure}
        \includegraphics[width=.8\textwidth]{./Seattle_logo_landscape_blue-black.pdf}
      \end{figure}
    \end{column}
  \end{columns}
\end{frame}

\subsection{Jeu de données}
\begin{frame}{\insertsubsection}
  \begin{itemize}
    \item Base de données issue de l'initiative de la ville de Seattle de proposer ses
          données en accès libre (Open Data)
    \item[]
    \item Données concernant les batiments de la ville, caractérise :
          \begin{itemize}
            \item le type,
            \item la surface,
            \item le nombre d'étages,
            \item la consomation énergétique,
            \item les émissions de carbone,
            \item $\vdots$
          \end{itemize}
    \item[]
    \item Données des années 2015 et 2016
  \end{itemize}
\end{frame}


\end{document}