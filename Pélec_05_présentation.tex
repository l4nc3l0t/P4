\documentclass[8pt,aspectratio=169,hyperref={unicode=true}]{beamer}

\usefonttheme{serif}
\usepackage{fontspec}
	\setmainfont{TeX Gyre Heros}
\usepackage{unicode-math}
\usepackage{lualatex-math}
	\setmathfont{TeX Gyre Termes Math}
\usepackage{polyglossia}
\setdefaultlanguage[frenchpart=false]{french}
\setotherlanguage{english}
%\usepackage{microtype}
\usepackage[locale = FR,
            separate-uncertainty,
            multi-part-units = single,
            range-units = single]{siunitx}
	\DeclareSIUnit\an{an}
  \DeclareSIUnit{\octet}{o}
\usepackage{amsmath}
\usepackage{amsfonts}
\usepackage{amssymb}
\usepackage{array}
\usepackage{graphicx}
\graphicspath{{./Figures/}}
\usepackage{booktabs}
\usepackage{tabularx}
\usepackage{multirow}
\usepackage{multicol}
    \newcolumntype{L}{>{\raggedright\arraybackslash}X}
    \newcolumntype{R}{>{\raggedleft\arraybackslash}X}
\usepackage{tikz}
\usetikzlibrary{graphs, graphdrawing, arrows.meta} \usegdlibrary{layered, trees}
\usetikzlibrary{overlay-beamer-styles}
\usepackage{subcaption}
\usepackage[]{animate}
\usepackage{float}
\usepackage{csquotes}

\usetheme[secheader
         ]{Boadilla}
\usecolortheme{seagull}
\setbeamertemplate{enumerate items}[default]
\setbeamertemplate{itemize items}{-}
\setbeamertemplate{navigation symbols}{}
\setbeamertemplate{bibliography item}{}
\setbeamerfont{framesubtitle}{size=\large}
\setbeamertemplate{section in toc}[sections numbered]
%\setbeamertemplate{subsection in toc}[subsections numbered]

\title[Anticipez les besoins en consommation électrique de bâtiments]
{Projet 4 : Anticipez les besoins en consommation électrique de bâtiments}
\author[Lancelot \textsc{Leclercq}]{Lancelot \textsc{Leclercq}} 
\institute[]{}
\date[]{\small{15 décembre 2021}}

\AtBeginSection[]{
  \begin{frame}
  \vfill
  \centering
    \usebeamerfont{title}\insertsectionhead\par%
  \vfill
  \end{frame}
}

\begin{document}
\setbeamercolor{background canvas}{bg=gray!20}
\begin{frame}[plain]
  \titlepage
\end{frame}

\begin{frame}{Sommaire}
  \Large
  \begin{columns}
    \begin{column}{.7\textwidth}
      \tableofcontents[hideallsubsections]
    \end{column}
  \end{columns}
\end{frame}


\section{Introduction}
\subsection{Problématique}
\begin{frame}{\insertsubsection}
  \begin{columns}
    \begin{column}{.6\textwidth}
      \begin{itemize}
        \item Objectif de la ville de Seattle : atteindre la neutralité en émissions
              de carbone
        \item[]
        \item La ville s'intéresse aux émissions des batiments non destinés
              à l'habitation
        \item[]
        \item Pour cela des relevés de consommation ont été réalisés mais ils sont
              couteux à obtenir
        \item[]
        \item Est-il possible de prédire les émissions et de la consommation d'énergie
              pour des batiments pour lesquels les relevés n'ont pas été réalisé à partir
              des relevés déjà obtenus
      \end{itemize}
    \end{column}
    \begin{column}{.4\textwidth}
      \begin{figure}
        \includegraphics[width=.8\textwidth]{./Seattle_logo_landscape_blue-black.pdf}
      \end{figure}
    \end{column}
  \end{columns}
\end{frame}

\subsection{Jeu de données}
\begin{frame}{\insertsubsection}
  \begin{itemize}
    \item Base de données issue de l'initiative de la ville de Seattle de proposer ses
          données en accès libre (Open Data)
    \item[]
    \item Données concernant les batiments de la ville, caractérise :
          \begin{itemize}
            \item le type,
            \item la surface,
            \item le nombre d'étages,
            \item la consomation énergétique,
            \item les émissions de carbone,
            \item $\vdots$
          \end{itemize}
    \item[]
    \item Données des années 2015 et 2016
  \end{itemize}
\end{frame}

\section{Nettoyage du jeu de données}
\subsection{Correction et selection des données}
\begin{frame}{\insertsection : \insertsubsection}
  \begin{columns}[t]
    \begin{column}{.5\textwidth}
      \includegraphics[width=\textwidth]{DataNbFull.pdf}
      \begin{itemize}
        \item Nettoyage des valeurs négative pour la surface des batiments/parkings,
              la consommation et les émissions
        \item Correction du nombre de d'étages aberrant pour certains batiments
        \item Lorsque le nombre de batiment est nul on remplace par 1
      \end{itemize}
    \end{column}
    \begin{column}{.5\textwidth}
      \includegraphics[width=\textwidth]{DataNbDrop.pdf}
      \begin{itemize}
        \item Suppression des batiments d'habitation
        \item Suppression des variables ayant moins de 50\% de données
        \item Suppressions des variables étant des relevés afin de voir si notre modèle
              peut s'en passer
      \end{itemize}
    \end{column}
  \end{columns}
\end{frame}

\subsection{Selections des variables}
\subsubsection{Élimination récursive des variables (RFE) et matrice de corrélation}
\begin{frame}{\insertsection : \insertsubsection}{\insertsubsubsection}
  \begin{columns}[t]
    \begin{column}{.5\textwidth}
      \centering
      Variables pertinentes pour les émissions
      \includegraphics[width=\textwidth]{RFECVEmissions.pdf}
    \end{column}
    \begin{column}{.5\textwidth}
      \centering
      Variables pertinentes pour la consommation
      \includegraphics[width=\textwidth]{RFECVConso.pdf}
    \end{column}
  \end{columns}
  \begin{itemize}
    \item Selection des variables les plus pertinentes par elimination recursive des variables (RFE)
    \item Réduction efficace pour les émissions
    \item Pas de réel changement de RMSE pour la consommation
  \end{itemize}
\end{frame}

\begin{frame}{\insertsection : \insertsubsection}{\insertsubsubsection}
  \begin{columns}[t]
    \begin{column}{.33\textwidth}
      \centering
      Variables pertinentes pour les émissions
      \includegraphics[width=\textwidth]{HeatmapUsedNumEmissions.pdf}
    \end{column}
    \begin{column}{.33\textwidth}
      \begin{itemize}
        \item Observation des résultats de RFE par les matrices de corrélation
        \item[]
        \item Les variables les plus corrélées sont communes aux deux sélection
        \item[]
        \item Conservation de 6 variables jugées pertinentes
      \end{itemize}
    \end{column}
    \begin{column}{.33\textwidth}
      \centering
      Variables pertinentes pour la consommation
      \includegraphics[width=\textwidth]{HeatmapUsedNumConso.pdf}
    \end{column}
  \end{columns}
\end{frame}

\subsubsection{Analyse en composantes principales (PCA)}
\begin{frame}[t]{\insertsection : \insertsubsection}{\insertsubsubsection}
  \begin{columns}
    \begin{column}{.33\textwidth}
      \includegraphics[width=\textwidth]{ScreePlot.pdf}
      \begin{itemize}
        \item Le graphique de la variance expliquée cumulée nous montre que
              99\% de la matrice est exliquée avec 5 variables
        \item Les quatres variables les plus corrélées se retrouvent sur l'axe F1
        \item L'EnergyStar score semble avoir une certaine importance car il explique
              une grande partie de l'axe F3
      \end{itemize}
    \end{column}
    \begin{column}{.33\textwidth}
      \centering
      \includegraphics[width=\textwidth]{PCAF1F2.pdf}
    \end{column}
    \begin{column}{.33\textwidth}
      \centering
      \includegraphics[width=\textwidth]{PCAF1F3.pdf}
    \end{column}
  \end{columns}
\end{frame}

\section{Étapes des modélisations}
\begin{frame}{\insertsection}
  \begin{columns}
    \begin{column}{.5\textwidth}
      Afin de comparer les différents modèles
      \begin{itemize}
        \item Split commun à chaque modèle (varie selon la variable modélisée)
        \item Pour chaque modèle (boucle) :
              \begin{itemize}
                \item GridSearch des meilleurs paramètres avec validation croisée
                \item Création d'un pipeline : scaling et fit du modèle
                      \begin{itemize}
                        \item Scaling par RobustScaler car plus résistant aux valeurs aberrantes
                              selon la documentation
                      \end{itemize}
              \end{itemize}
      \end{itemize}
    \end{column}
    \begin{column}{.5\textwidth}
      \begin{itemize}
        \item La boucle retourne :
              \begin{itemize}
                \item Le(s) meilleur(s) paramètre(s) (gridsearch)
                \item La RMSE en fonction du paramètre le plus évolutif (validation croisée)
                \item La figure de la variable étudiée vs ses prédictions
                \item Le R², la RMSE, la MAE (mean absolute error) et le temps de calcul du modèle
              \end{itemize}
      \end{itemize}
    \end{column}
  \end{columns}
  \vspace{.5cm}
  \centering
  \tikz [rounded corners, every node/.style={anchor=west}, level sep = 5mm, >={Stealth}]
  \graph [layered layout, grow=right, nodes={draw, font=\footnotesize}, head anchor=west, tail anchor=east,
  edges=rounded corners, sibling distance=5mm]{
  Data -> train test split  -> Cross Validation -> {Split 5, Split 4, Split 3, Split 2, Split 1}
  -- Pipeline [draw] // {RobustScaler -> "Modèle.fit"[align here]},
  "GridSearch : Recherche des meilleurs paramètres"  [draw]  // {Cross Validation, Split 5, Split 4,
  Split 3, Split 2, Split 1, RobustScaler, "Modèle.fit"}
  -> Résultats [draw] // { Meilleur paramètre [grow=down] --[draw=none] Figure RMSE --[draw=none]
  Figure variable test vs variable prédite[align here] --[draw=none] "R² RMSE, MAE et temps de calcul"}
  };

\end{frame}

\section[Modélisation émissions]{Modélisation des émissions de carbone}
\subsection{Modèle Ridge}
\begin{frame}{\insertsubsection}
  \begin{columns}[t]
    \begin{column}{.33\textwidth}
      \centering Variable non modifiée
      \includegraphics[width=.97\textwidth]{EmissionsGraphRMSERidge.pdf}
      \includegraphics[width=.97\textwidth]{EmissionsTestvsPredRidge.pdf}
    \end{column}
    \begin{column}{.33\textwidth}
      $\Longleftarrow$

      \input{./Tableaux/EmissionsBestParamsRidge.tex}

      \raggedright
      $\Longleftarrow$

      \raggedleft
      $\Longrightarrow$

      \input{./Tableaux/EmissionsBestParamsRidge_log.tex}

      \raggedleft
      $\Longrightarrow$

      \raggedright
      \begin{itemize}
        \item Modèle de régression linéaire introduisant un coefficient cherchant à minimiser
              l'erreur quadratique
      \end{itemize}

      $\Longleftarrow$
      {\scriptsize \centering
          \input{./Tableaux/EmissionsScoreRidge.tex}}

      $\Longleftarrow$

      \raggedleft
      $\Longrightarrow$

      {\scriptsize \centering
          \input{./Tableaux/EmissionsScoreRidge_log.tex}
        }
      \raggedleft
      $\Longrightarrow$
    \end{column}
    \begin{column}{.33\textwidth}
      \centering Variable au log
      \includegraphics[width=.96\textwidth]{EmissionsGraphRMSERidge_log.pdf}
      \includegraphics[width=.96\textwidth]{EmissionsTestvsPredRidge_log.pdf}
    \end{column}
  \end{columns}
\end{frame}

\subsection{Modèle Lasso}
\begin{frame}{\insertsubsection}
  \begin{columns}[t]
    \begin{column}{.33\textwidth}
      \centering Variable non modifiée
      \includegraphics[width=.97\textwidth]{EmissionsGraphRMSELasso.pdf}
      \includegraphics[width=.97\textwidth]{EmissionsTestvsPredLasso.pdf}
    \end{column}
    \begin{column}{.33\textwidth}
      $\Longleftarrow$

      {\footnotesize
          \input{./Tableaux/EmissionsBestParamsLasso.tex}}

      \raggedright
      $\Longleftarrow$

      \raggedleft
      $\Longrightarrow$

      {\footnotesize
          \input{./Tableaux/EmissionsBestParamsLasso_log.tex}}

      \raggedleft
      $\Longrightarrow$

      \raggedright
      \begin{itemize}
        \item Similaire à la regression ridge
        \item Coefficient est réduit à zéro pour les variables peu corrélées
        \item Peut être utilisé pour la sélection de feature
      \end{itemize}

      $\Longleftarrow$
      {\scriptsize \centering
          \input{./Tableaux/EmissionsScoreLasso.tex}}

      $\Longleftarrow$

      \raggedleft
      $\Longrightarrow$

      {\scriptsize \centering
          \input{./Tableaux/EmissionsScoreLasso_log.tex}
        }
      \raggedleft
      $\Longrightarrow$

    \end{column}
    \begin{column}{.33\textwidth}
      \centering Variable au log
      \includegraphics[width=.96\textwidth]{EmissionsGraphRMSELasso_log.pdf}
      \includegraphics[width=.96\textwidth]{EmissionsTestvsPredLasso_log.pdf}
    \end{column}
  \end{columns}
\end{frame}

\subsection{Modèle ElasticNet}
\begin{frame}{\insertsubsection}
  \begin{columns}[t]
    \begin{column}{.33\textwidth}
      \centering Variable non modifiée
      \includegraphics[width=.97\textwidth]{EmissionsGraphRMSEElasticNet.pdf}
      \includegraphics[width=.97\textwidth]{EmissionsTestvsPredElasticNet.pdf}
    \end{column}
    \begin{column}{.33\textwidth}
      $\Longleftarrow$

      {\footnotesize
      \input{./Tableaux/EmissionsBestParamsElasticNet.tex}}

      \raggedright
      $\Longleftarrow$

      \raggedleft
      $\Longrightarrow$

      {\footnotesize
      \input{./Tableaux/EmissionsBestParamsElasticNet_log.tex}}

      \raggedleft
      $\Longrightarrow$

      \raggedright
      \begin{itemize}
        \item Combine les coefficients des regressions ridge et lasso
      \end{itemize}

      $\Longleftarrow$
      {\scriptsize \centering
          \input{./Tableaux/EmissionsScoreElasticNet.tex}}

      $\Longleftarrow$

      \raggedleft
      $\Longrightarrow$

      {\scriptsize \centering
          \input{./Tableaux/EmissionsScoreElasticNet_log.tex}
        }
      \raggedleft
      $\Longrightarrow$
    \end{column}
    \begin{column}{.33\textwidth}
      \centering Variable au log
      \includegraphics[width=.96\textwidth]{EmissionsGraphRMSEElasticNet_log.pdf}
      \includegraphics[width=.96\textwidth]{EmissionsTestvsPredElasticNet_log.pdf}
    \end{column}
  \end{columns}
\end{frame}

\subsection{Modèle kNeighborsRegressor}
\begin{frame}{\insertsubsection}
  \begin{columns}[t]
    \begin{column}{.33\textwidth}
      \centering Variable non modifiée
      \includegraphics[width=.97\textwidth]{EmissionsGraphRMSEKNeighborsRegressor.pdf}
      \includegraphics[width=.97\textwidth]{EmissionsTestvsPredKNeighborsRegressor.pdf}
    \end{column}
    \begin{column}{.33\textwidth}
       $\Longleftarrow$

      {\footnotesize
      \input{./Tableaux/EmissionsBestParamsKNeighborsRegressor.tex}}

      \raggedright
      $\Longleftarrow$

      \raggedleft
      $\Longrightarrow$
      
      {\footnotesize
      \input{./Tableaux/EmissionsBestParamsKNeighborsRegressor_log.tex}}

      \raggedleft
      $\Longrightarrow$

      \raggedright
      \begin{itemize}
        \item Prédiction par interpolation avec les plus proches voisins dans le jeu de données
      \end{itemize}

      $\Longleftarrow$
      {\scriptsize \centering
          \input{./Tableaux/EmissionsScoreKNeighborsRegressor.tex}}

      $\Longleftarrow$

      \raggedleft
      $\Longrightarrow$

      {\scriptsize \centering
          \input{./Tableaux/EmissionsScoreKNeighborsRegressor_log.tex}
        }
      \raggedleft
      $\Longrightarrow$
    \end{column}
    \begin{column}{.33\textwidth}
      \centering Variable au log
      \includegraphics[width=.96\textwidth]{EmissionsGraphRMSEKNeighborsRegressor_log.pdf}
      \includegraphics[width=.96\textwidth]{EmissionsTestvsPredKNeighborsRegressor_log.pdf}
    \end{column}
  \end{columns}
\end{frame}

\subsection{Modèle RandomForestRegressor}
\begin{frame}{\insertsubsection}
  \begin{columns}[t]
    \begin{column}{.33\textwidth}
      \centering Variable non modifiée
      \includegraphics[width=.97\textwidth]{EmissionsGraphRMSERandomForestRegressor.pdf}
      \includegraphics[width=.97\textwidth]{EmissionsTestvsPredRandomForestRegressor.pdf}
    \end{column}
    \begin{column}{.33\textwidth}
      $\Longleftarrow$

      {\footnotesize
      \input{./Tableaux/EmissionsBestParamsRandomForestRegressor.tex}}

      \raggedright
      $\Longleftarrow$

      \raggedleft
      $\Longrightarrow$
      
      {\footnotesize
      \input{./Tableaux/EmissionsBestParamsRandomForestRegressor_log.tex}}

      \raggedleft
      $\Longrightarrow$

      \raggedright
      \begin{itemize}
        \item Classification des valeurs à partir d'arbre de décision aléatoire
        \item Prédiction à partir de ces classifieurs
      \end{itemize}

      $\Longleftarrow$
      {\scriptsize \centering
          \input{./Tableaux/EmissionsScoreRandomForestRegressor.tex}}

      $\Longleftarrow$

      \raggedleft
      $\Longrightarrow$

      {\scriptsize \centering
          \input{./Tableaux/EmissionsScoreRandomForestRegressor_log.tex}
        }
      \raggedleft
      $\Longrightarrow$
    \end{column}
    \begin{column}{.33\textwidth}
      \centering Variable au log
      \includegraphics[width=.96\textwidth]{EmissionsGraphRMSERandomForestRegressor_log.pdf}
      \includegraphics[width=.96\textwidth]{EmissionsTestvsPredRandomForestRegressor_log.pdf}
    \end{column}
  \end{columns}
\end{frame}

\subsection{Modèle AdaBoostRegressor}
\begin{frame}{\insertsubsection}
  \begin{columns}[t]
    \begin{column}{.33\textwidth}
      \centering Variable non modifiée
      \includegraphics[width=.97\textwidth]{EmissionsGraphRMSEAdaBoostRegressor.pdf}
      \includegraphics[width=.97\textwidth]{EmissionsTestvsPredAdaBoostRegressor.pdf}
    \end{column}
    \begin{column}{.33\textwidth}
      $\Longleftarrow$

      {\tiny
      \input{./Tableaux/EmissionsBestParamsAdaBoostRegressor.tex}}

      $\Longleftarrow$ \hfill $\Longrightarrow$
      
      \raggedleft
      {\tiny
      \input{./Tableaux/EmissionsBestParamsAdaBoostRegressor_log.tex}}

      $\Longrightarrow$

      \scriptsize
      \begin{itemize}
        \item Même principe que les forêts aléatoires
        \item Utilisation d'apprenants faibles (légèrement plus performants que la prediction aléatoire
              similaire à de petits arbre de décision)
        \item Les prédictions des apprenants sont combinées avec un coefficient de poids
        \item À chaque itération le poids des mauvaises prédictions est augmenté ce qui pousse le modèle
              à se concentrer dessus
      \end{itemize}

      \normalsize
      \raggedright
      $\Longleftarrow$

      {\tiny
          \input{./Tableaux/EmissionsScoreAdaBoostRegressor.tex}}

      $\Longleftarrow$ \hfill $\Longrightarrow$

      \raggedleft
      {\tiny
          \input{./Tableaux/EmissionsScoreAdaBoostRegressor_log.tex}
        }

      $\Longrightarrow$
    \end{column}
    \begin{column}{.33\textwidth}
      \centering Variable au log
      \includegraphics[width=.96\textwidth]{EmissionsGraphRMSEAdaBoostRegressor_log.pdf}
      \includegraphics[width=.96\textwidth]{EmissionsTestvsPredAdaBoostRegressor_log.pdf}
    \end{column}
  \end{columns}
\end{frame}

\subsection{Modèle GradientBoostingRegressor}
\begin{frame}{\insertsubsection}
  \begin{columns}[t]
    \begin{column}{.33\textwidth}
      \centering Variable non modifiée
      \includegraphics[width=.97\textwidth]{EmissionsGraphRMSEGradientBoostingRegressor.pdf}
      \includegraphics[width=.97\textwidth]{EmissionsTestvsPredGradientBoostingRegressor.pdf}
    \end{column}
    \begin{column}{.33\textwidth}
      $\Longleftarrow$

      {\footnotesize
      \input{./Tableaux/EmissionsBestParamsGradientBoostingRegressor.tex}}

      $\Longleftarrow$ \hfill $\Longrightarrow$
      
      \raggedleft
      {\footnotesize 
      \input{./Tableaux/EmissionsBestParamsGradientBoostingRegressor_log.tex}}

      $\Longrightarrow$

      \raggedright
      \begin{itemize}
        \item Similaire à AdaBoostRegressor
        \item Prend en compte une fonction objectif (loss fonction) plus complexe afin
              d'améliorer l'optimisation
      \end{itemize}

      $\Longleftarrow$
      {\scriptsize \centering
          \input{./Tableaux/EmissionsScoreGradientBoostingRegressor.tex}}

      $\Longleftarrow$

      \raggedleft
      $\Longrightarrow$

      {\scriptsize \centering
          \input{./Tableaux/EmissionsScoreGradientBoostingRegressor_log.tex}
        }
      \raggedleft
      $\Longrightarrow$
    \end{column}
    \begin{column}{.33\textwidth}
      \centering Variable au log
      \includegraphics[width=.96\textwidth]{EmissionsGraphRMSEGradientBoostingRegressor_log.pdf}
      \includegraphics[width=.96\textwidth]{EmissionsTestvsPredGradientBoostingRegressor_log.pdf}
    \end{column}
  \end{columns}
\end{frame}

\subsection{Comparaison des résultats selon que la variable est au log ou non}
\begin{frame}{\insertsubsection}
  \begin{columns}
    \begin{column}{.5\textwidth}
      \centering
      \includegraphics[width=.95\textwidth]{EmissionsCompareScores.pdf}
    \end{column}
    \begin{column}{.5\textwidth}
      \begin{itemize}
        \item RandomForestRegressor, AdaBoostRegressor et GradientBoostingRegressor ont
              des erreur moins importantes et un R² plus grand quelque soit la variable modélisée
        \item[]
        \item KNeighborsRegressor est plus performant avec la variable au log
        \item[]
        \item Modèles linéaire : Ridge, Lasso et ElasticNet moins efficaces avec la variable au log
        \item[]
        \item Temps de modélisation de RandomForestRegressor et GradientBoostingRegressor
              plus importants que les autres
        \item[]
        \item Temps de modélisation de RandomForestRegressor avec la variable au log moindre
              qu'avec la variable non modifiée
      \end{itemize}
    \end{column}
  \end{columns}
\end{frame}

\subsection{Influence de l'EnergyStar score sur la prédiction des Émissions}
\begin{frame}{\insertsubsection}
  \begin{columns}
    \begin{column}{.6\textwidth}
      \centering
      \includegraphics[width=\textwidth]{EmissionsCompareScoresESS.pdf}
    \end{column}
    \begin{column}{.4\textwidth}
      \begin{itemize}
        \item GradientBoostingRegressor avec la variable au log (RMSE la plus petite)
        \item[]
        \item L'EnergyStar score améliore la RMSE
        \item[]
        \item Amélioration des les autres mesures d'erreur et de corrélation
      \end{itemize}
    \end{column}
  \end{columns}
\end{frame}

\section[Modélisation consommation]{Modélisation de la consommation énergétique}
\subsection{Modèle Ridge}
\begin{frame}{\insertsubsection}
  \begin{columns}[t]
    \begin{column}{.31\textwidth}
      \centering Variable non modifiée
      \includegraphics[width=\textwidth]{ConsoGraphRMSERidge.pdf}
      \includegraphics[width=\textwidth]{ConsoTestvsPredRidge.pdf}
    \end{column}
    \begin{column}{.38\textwidth}
      $\Longleftarrow$

      \scriptsize
      {\centering
        \input{./Tableaux/ConsoScoreRidge.tex}}
      \input{./Tableaux/ConsoBestParamsRidge.tex}

      \normalsize
      $\Longleftarrow$

      \raggedleft $\Longrightarrow$

      \scriptsize
      {\centering
        \input{./Tableaux/ConsoScoreRidge_log.tex}}
      \input{./Tableaux/ConsoBestParamsRidge_log.tex}

      \normalsize
      $\Longrightarrow$
    \end{column}
    \begin{column}{.31\textwidth}
      \centering Variable au log
      \includegraphics[width=\textwidth]{ConsoGraphRMSERidge_log.pdf}
      \includegraphics[width=\textwidth]{ConsoTestvsPredRidge_log.pdf}
    \end{column}
  \end{columns}
\end{frame}

\subsection{Modèle Lasso}
\begin{frame}{\insertsubsection}
  \begin{columns}[t]
    \begin{column}{.31\textwidth}
      \centering Variable non modifiée
      \includegraphics[width=\textwidth]{ConsoGraphRMSELasso.pdf}
      \includegraphics[width=\textwidth]{ConsoTestvsPredLasso.pdf}
    \end{column}
    \begin{column}{.38\textwidth}
      $\Longleftarrow$
      \scriptsize
      {\centering
        \input{./Tableaux/ConsoScoreLasso.tex}}
      \input{./Tableaux/ConsoBestParamsLasso.tex}

      \normalsize
      $\Longleftarrow$

      \raggedleft $\Longrightarrow$
      \scriptsize
      {\centering
        \input{./Tableaux/ConsoScoreLasso_log.tex}}
      \input{./Tableaux/ConsoBestParamsLasso_log.tex}

      \normalsize
      $\Longrightarrow$
    \end{column}
    \begin{column}{.31\textwidth}
      \centering Variable au log
      \includegraphics[width=\textwidth]{ConsoGraphRMSELasso_log.pdf}
      \includegraphics[width=\textwidth]{ConsoTestvsPredLasso_log.pdf}
    \end{column}
  \end{columns}
\end{frame}

\subsection{Modèle ElasticNet}
\begin{frame}{\insertsubsection}
  \begin{columns}[t]
    \begin{column}{.31\textwidth}
      \centering Variable non modifiée
      \includegraphics[width=\textwidth]{ConsoGraphRMSEElasticNet.pdf}
      \includegraphics[width=\textwidth]{ConsoTestvsPredElasticNet.pdf}
    \end{column}
    \begin{column}{.38\textwidth}
      $\Longleftarrow$
      \scriptsize
      {\centering
        \input{./Tableaux/ConsoScoreElasticNet.tex}}
      \input{./Tableaux/ConsoBestParamsElasticNet.tex}

      \normalsize
      $\Longleftarrow$

      \raggedleft $\Longrightarrow$
      \scriptsize
      {\centering
        \input{./Tableaux/ConsoScoreElasticNet_log.tex}}
      \input{./Tableaux/ConsoBestParamsElasticNet_log.tex}

      \normalsize
      $\Longrightarrow$
    \end{column}
    \begin{column}{.31\textwidth}
      \centering Variable au log
      \includegraphics[width=\textwidth]{ConsoGraphRMSEElasticNet_log.pdf}
      \includegraphics[width=\textwidth]{ConsoTestvsPredElasticNet_log.pdf}
    \end{column}
  \end{columns}
\end{frame}

\subsection{Modèle kNeighborsRegressor}
\begin{frame}{\insertsubsection}
  \begin{columns}[t]
    \begin{column}{.31\textwidth}
      \centering Variable non modifiée
      \includegraphics[width=\textwidth]{ConsoGraphRMSEKNeighborsRegressor.pdf}
      \includegraphics[width=\textwidth]{ConsoTestvsPredKNeighborsRegressor.pdf}
    \end{column}
    \begin{column}{.38\textwidth}
      $\Longleftarrow$
      \scriptsize
      {\centering
        \input{./Tableaux/ConsoScoreKNeighborsRegressor.tex}}
      \input{./Tableaux/ConsoBestParamsKNeighborsRegressor.tex}

      \normalsize
      $\Longleftarrow$

      \raggedleft $\Longrightarrow$
      \scriptsize
      {\centering
        \input{./Tableaux/ConsoScoreKNeighborsRegressor_log.tex}}
      \input{./Tableaux/ConsoBestParamsKNeighborsRegressor_log.tex}

      \normalsize
      $\Longrightarrow$
    \end{column}
    \begin{column}{.31\textwidth}
      \centering Variable au log
      \includegraphics[width=\textwidth]{ConsoGraphRMSEKNeighborsRegressor_log.pdf}
      \includegraphics[width=\textwidth]{ConsoTestvsPredKNeighborsRegressor_log.pdf}
    \end{column}
  \end{columns}
\end{frame}

\subsection{Modèle RandomForestRegressor}
\begin{frame}{\insertsubsection}
  \begin{columns}[t]
    \begin{column}{.31\textwidth}
      \centering Variable non modifiée
      \includegraphics[width=\textwidth]{ConsoGraphRMSERandomForestRegressor.pdf}
      \includegraphics[width=\textwidth]{ConsoTestvsPredRandomForestRegressor.pdf}
    \end{column}
    \begin{column}{.38\textwidth}
      $\Longleftarrow$
      \scriptsize
      {\centering
        \input{./Tableaux/ConsoScoreRandomForestRegressor.tex}}
      \input{./Tableaux/ConsoBestParamsRandomForestRegressor.tex}

      \normalsize
      $\Longleftarrow$

      \raggedleft $\Longrightarrow$
      \scriptsize
      {\centering
        \input{./Tableaux/ConsoScoreRandomForestRegressor_log.tex}}
      \input{./Tableaux/ConsoBestParamsRandomForestRegressor_log.tex}

      \normalsize
      $\Longrightarrow$
    \end{column}
    \begin{column}{.31\textwidth}
      \centering Variable au log
      \includegraphics[width=\textwidth]{ConsoGraphRMSERandomForestRegressor_log.pdf}
      \includegraphics[width=\textwidth]{ConsoTestvsPredRandomForestRegressor_log.pdf}
    \end{column}
  \end{columns}
\end{frame}

\subsection{Modèle AdaBoostRegressor}
\begin{frame}{\insertsubsection}
  \begin{columns}[t]
    \begin{column}{.31\textwidth}
      \centering Variable non modifiée
      \includegraphics[width=\textwidth]{ConsoGraphRMSEAdaBoostRegressor.pdf}
      \includegraphics[width=\textwidth]{ConsoTestvsPredAdaBoostRegressor.pdf}
    \end{column}
    \begin{column}{.38\textwidth}
      $\Longleftarrow$
      \scriptsize
      {\centering
        \input{./Tableaux/ConsoScoreAdaBoostRegressor.tex}}
      \input{./Tableaux/ConsoBestParamsAdaBoostRegressor.tex}

      \normalsize
      $\Longleftarrow$

      \raggedleft $\Longrightarrow$
      \scriptsize
      {\centering
        \input{./Tableaux/ConsoScoreAdaBoostRegressor_log.tex}}
      \input{./Tableaux/ConsoBestParamsAdaBoostRegressor_log.tex}

      \normalsize
      $\Longrightarrow$
    \end{column}
    \begin{column}{.31\textwidth}
      \centering Variable au log
      \includegraphics[width=\textwidth]{ConsoGraphRMSEAdaBoostRegressor_log.pdf}
      \includegraphics[width=\textwidth]{ConsoTestvsPredAdaBoostRegressor_log.pdf}
    \end{column}
  \end{columns}
\end{frame}

\subsection{Modèle GradientBoostingRegressor}
\begin{frame}{\insertsubsection}
  \begin{columns}[t]
    \begin{column}{.31\textwidth}
      \centering Variable non modifiée
      \includegraphics[width=\textwidth]{ConsoGraphRMSEGradientBoostingRegressor.pdf}
      \includegraphics[width=\textwidth]{ConsoTestvsPredGradientBoostingRegressor.pdf}
    \end{column}
    \begin{column}{.38\textwidth}
      $\Longleftarrow$
      \scriptsize
      {\centering
        \input{./Tableaux/ConsoScoreGradientBoostingRegressor.tex}}
      \input{./Tableaux/ConsoBestParamsGradientBoostingRegressor.tex}

      \normalsize
      $\Longleftarrow$

      \raggedleft $\Longrightarrow$
      \scriptsize
      {\centering
        \input{./Tableaux/ConsoScoreGradientBoostingRegressor_log.tex}}
      \input{./Tableaux/ConsoBestParamsGradientBoostingRegressor_log.tex}

      \normalsize
      $\Longrightarrow$
    \end{column}
    \begin{column}{.31\textwidth}
      \centering Variable au log
      \includegraphics[width=\textwidth]{ConsoGraphRMSEGradientBoostingRegressor_log.pdf}
      \includegraphics[width=\textwidth]{ConsoTestvsPredGradientBoostingRegressor_log.pdf}
    \end{column}
  \end{columns}
\end{frame}

\subsection{Comparaison des résultats selon que la variable est au log ou non}
\begin{frame}{\insertsubsection}
  \begin{columns}
    \begin{column}{.5\textwidth}
      \centering
      \includegraphics[width=.95\textwidth]{ConsoCompareScores.pdf}
    \end{column}
    \begin{column}{.5\textwidth}
      \begin{itemize}
        \item RMSE KNeighborsRegressor, RandomForestRegressor, AdaBoostRegressor et
              GradientBoostingRegressor inférieures avec la variable au log
        \item[]
        \item RMSE de RandomForestRegressor et GradientBoostingRegressor légèrement
              inférieures quelque soit la variable
        \item[]
        \item MAE de RandomForestRegressor et GradientBoostingRegressor plus significativement
              inférieures quelque soit la variable
        \item[]
        \item Temps de modélisation plus important pour GradientBoostingRegressor
      \end{itemize}
    \end{column}
  \end{columns}
\end{frame}

\section{Conclusion}
\begin{frame}{\insertsection}
  \begin{itemize}
    \item Découverte des différents modèles et de leur fonctionnement
    \item[]
    \item Obtention avec certains modèles d'une estimation avec moins de 1\% d'écart
          à la moyenne absolue
    \item[]
    \item Si de nouveaux batiments ont été construits il peut être intéressant de rentrer leurs
          caractéristiques dans notre base de donnée et voir si on peut prédire leurs émissions
          et consommation quitte à faire des mesures pour estimer si ces prédictions sont bonnes
  \end{itemize}
\end{frame}

\end{document}